% Randy Westlund's Personal CV

\documentclass[10pt,letterpaper]{article}
\usepackage[letterpaper,margin=0.75in]{geometry}
\usepackage[utf8]{inputenc}
\usepackage{mdwlist}
\usepackage[T1]{fontenc}
\usepackage{textcomp}
\usepackage{tgpagella}
\usepackage{gensymb}
\usepackage{enumitem}
\usepackage{tabularx}
% No space above enumerate.
\setlist{nolistsep}
\pagestyle{empty}
\setlength{\tabcolsep}{0em}
\setlist[itemize]{itemsep=0.2em}

% Format two pieces of text; one left aligned and one right aligned.
\newcommand{\headerrow}[2]{%
    \begin{tabularx}{\linewidth}{Xr}
	    #1 & #2 \\
    \end{tabularx}
}

\begin{document}
\begin{center}
{\LARGE \textbf{Randy J. Westlund, E.I.}}

28 Princeton St Apt 2 \ \ \textbullet%
    \ \ Medford, MA 02155 \\
(704) 526 5337 \ \ \textbullet%
    \ \ rwestlun@gmail.com \ \ \textbullet%
    \ \ www.textplain.net
\end{center}

\hrule
\vspace{-0.4em}
\subsection*{Education}
\begin{itemize*}
	\parskip=0.1em
	\item
	\headerrow{\textbf{The University of North Carolina at Charlotte}}
		{\textbf{Charlotte, NC}}
	\headerrow{\emph{B.S., Computer Engineering}}{\emph{2011}}
		\emph{Minor: Mathematics, Classical Studies} \\
        \emph{Major GPA\@: 3.70, Overall GPA\@: 3.56} \\
        \textbf{Engineer Intern},  FE Certificate Number A-26823 \\
        \textbf{Senior Design Project}
        \begin{itemize*}
            \item Won second place of 50 teams, leading 5 engineers to design
                and build an autonomous robot with obstacle avoidance and a
                user-controlled videoconference mode.  Cross compiled Linux for
                an embedded ARM board, and integrated computer vision,
                ultrasonic, IR, and tactile sensors.
        \end{itemize*}
    \textbf{Course Projects}
    \begin{itemize*}
        \item Wrote a custom implementation of \texttt{malloc()/free()} %chktex 36
            and deployed it on a Renesas microcontroller.

        \item Implemented a parallelized, pipelined, FPGA version of the
            Needleman-Wunsch algorithm for global sequence alignment of A-C-G-T
            protein sequences and benchmarked it against a CPU implementation.

        \item Implemented 3D graphics rendering algorithms of Lambertian and
            specular surfaces with the Phong Reflection Model in MATLAB\@.

        \item Designed, simulated, and constructed a four-stage 60 dB amplifier
            with BJTs.

        \item Worked in teams to design and process RFID antennas on silicon
            wafers in a clean room.
    \end{itemize*}
\end{itemize*}

\hrule
\vspace{-0.4em}
\subsection*{Experience}
\begin{itemize}
	\parskip=0.1em

	\item
	\headerrow{\textbf{Awning Tracker}}{\textbf{Medford, MA}}
	\headerrow{\emph{Founder and CEO}}{\emph{July 2016--Present}}
	\begin{itemize}
        \item Founded a startup company to provide specialized project
            management software for the awning industry.

        \item Prototyped and deployed a beta version of Awning Tracker on a
            MEAN stack.

        \item Created the final Awning Tracker platform as a client-side web
            application using web components and the Polymer framework.
            Implemented the back end with PostgreSQL and Go.

        \item Managed a large number of Awning Tracker instances across
            multiple FreeBSD servers and jails, including replicating backups
            to multiple data centers.  Designed the architecture and wrote
            custom automation and monitoring tools.

        \item Created advertisements and ran marketing campaigns across various
            platforms.

        \item Met with clients and led in-person training seminars.  Provided
            24/7 support.

        \item Led business meetings with clients to study paper-driven
            workflows and discuss areas where technology could improve
            efficiency and communication.
	\end{itemize}

    \item
    \headerrow{\textbf{Experimeta, LLC}}{\textbf{Medford, MA}}
    \headerrow{\emph{Co-Founder and CEO}}{\emph{July 2015--Present}}
    \begin{itemize}
        \item Designed and implemented a second iteration of the firmware,
            control software, and communication protocol for an autonomous
            internet-connected germicidal ultraviolet light device (see SPI
            Team, below).  Implemented the new control software as a
            web-application with back-end database and bridge to the XBee
            network.  Repaired faulty hardware and made units ready for sale.
            Traveled to Kansas to demo and install the improved units.
    \end{itemize}

    \item
    \headerrow{\textbf{Earth, Atmospheric, and Planetary Sciences Department, MIT}}
        {\textbf{Cambridge, MA}}
    \headerrow{\emph{System Architect, Researcher}}{\emph{April 2014--Present}}
    \begin{itemize}
        \item Took the primary role in developing SLOOP, a pattern retrieval
            engine for animal biometrics.  SLOOP helps biologists identify
            individual members of a species from a collection of photographs to
            model population over time.

        \item Redesigned the system from the ground up, implementing a RESTful
            API using Node.js, and Angular.js for the new mobile-friendly
            client-side application and management console. Expanded the system
            to handle new species.  Improved reliability by migrating all data
            from the filesystem to a PostgreSQL database.  Packaged the system
            into a deployable virtual machine.

        \item Traveled solo to New Zealand to upgrade and relocate the
            Department of Conservation's existing SLOOP system.
    \end{itemize}

    \headerrow{\emph{Systems Engineer via E5 Aeronautics}}
        {\emph{July 2013--Present}}
    \begin{itemize}
        \item Designed embedded systems and communication protocols for small
            unmanned aircraft for civilian use.  Explored
            issues such as mesh networking, low-latency remote control, and
            redundant sensor systems.  Used MAVLink and custom communication
            protocols.

        \item Cross compiled and deployed a custom NanoBSD image to embedded
            boards to provide communications over 900 MHz.  Implemented
            realtime video and telemetry feeds from the aircraft to the ground.

        \item Modified firmware for Wi-Fi SD cards and Canon cameras with CHDK,
            wrote custom device drivers for 900 MHz radios, and deployed to an
            embedded Linux development board.  Utilized memory-mapped files to
            meet near realtime message passing requirements.

        \item Tested unmanned aerial systems (UAS/UAV) at Popocatépetl, an active
            volcano in Mexico. Flew aircraft through the volcano's plume to
            collect environmental data.
    \end{itemize}

    \item
    \headerrow{\textbf{Urban Scholars, University of Massachusetts Boston}}
		{\textbf{Boston, MA}}
	\headerrow{\emph{Robotics Instructor}}{\emph{June 2013--December 2013}}
	\begin{itemize}
        \item Designed and taught evening and summer courses on Beginning
            Robotics, Advanced Robotics, and Introductory Programming, for
            students aged 11--16, in line with the Common Core standards.
            Topics included Lego Mindstorms, Arduinos, gearboxes, LEDs,
            encryption and internet safely, robotics in the news, ethical
            considerations, and historical figures in computing.
	\end{itemize}

	\item
	\headerrow{\textbf{SPI Team, LLC}}{\textbf{Highland, MD}}
	\headerrow{\emph{Co-Founder and Lead Engineer}}
		{\emph{February 2012--June 2013}}
	\begin{itemize}
        \item Co-founded with my retired mentor from NASA and organized a team
            of engineers.

        \item Led a team that designed and implemented embedded electronics to
            control robotic arms with inverse kinematics with a team.
            Personally designed the circuit and wrote the I/O and
            communications code.

        \item Collaborated on market research and business strategy with a
            patent holder for an autonomous internet-connected germicidal
            ultraviolet light device, and led the prototype design team.
            Personally wrote the embedded firmware that drives the motors and
            various human occupancy sensors and designed a communication
            protocol for the 802.15.4 mesh XBee radios.  Deliverables included
            the prototype device, control software, and engineering
            documentation.
        % Removed to make Ann happy: EHS name and patent numbers 7692172 and 8097861.
	\end{itemize}

	\item
	\headerrow{\textbf{NASA Goddard Space Flight Center}}{\textbf{Greenbelt, MD}}
	\headerrow{\emph{Apprentice/Lead Intern, Robotics Engineering Boot Camp}}
		{\emph{June 2011--Aug 2011}}
	\begin{itemize}
        \item Designed and implemented a software-defined radio (SDR) using a
            Universal Software Radio Peripheral (USRP) with GNU Radio to
            receive, demodulate, and decode L-band and S-band satellite
            communications from NOAA's Polar Operational Environmental
            Satellites.

        \item Traveled to the NASA Antarctic Interactive Launch Support
            system at the NASA Satellite Ground Station in McMurdo, Antarctica
            and replaced on-site analog equipment with my SDR in December 2011.

        \item Returned to Antarctica in December 2012 to test system, record
            satellite data, and package the 28-year-old ground station for
            return to NASA Goddard.
	\end{itemize}

	\item
	\headerrow{\textbf{Charlotte Tent and Awning Co., Inc.}}{\textbf{Charlotte, NC}}
	\headerrow{\emph{Bookkeeper/Assistant}}{\emph{June 2005--September 2009}}
	\begin{itemize}
        \item Managed accounts receivable and accounts payable for company with
            \$2.5 million in revenue.

        \item Assisted in design, build, and setup of trade show displays for
            conventions of 20,000--30,000 attendees.
	\end{itemize}

\end{itemize}

\hrule
\vspace{-0.4em}
\subsection*{Core Technical Skills}
\begin{itemize*}
    \item \textbf{Programming Languages:} C, C\texttt{++}, C\#, Rust, Go, Java,
            Perl, Python, JavaScript
    \item \textbf{Domain-Specific:} VHDL, SQL, BSD Make, GNU Make, GDB, Shell
            Scripting, Regular Expressions, HTML/CSS, \LaTeX, Octave/MATLAB
    \item \textbf{Operating Systems}
        \begin{itemize*}
            \item FreeBSD, TrueOS, NanoBSD, FreeNAS, pfSense, Jails, ZFS, Poudriere
            \item OpenBSD, PF
            \item Gentoo Linux, Arch Linux, Ubuntu
        \end{itemize*}
    \item \textbf{System Administration}
        \begin{itemize*}
            \item Familiarity with server rooms
            \item RAID, PXE booting
            \item Apache, Nginx, PostgreSQL, MongoDB
        \end{itemize*}
    \item \textbf{Frameworks}
        \begin{itemize*}
            \item \textbf{Web:} Polymer, Django, Node.js, Angular.js, jQuery
            \item \textbf{Robotics:} Robot Operating System (ROS), MAVLink
            \item \textbf{Version Control:} Git, SVN, CVS
        \end{itemize*}
    \item \textbf{Embedded Systems \& Electronics}
        \begin{itemize*}
            \item Arduino, AVR, BeagleBone, misc embedded ARM boards
            \item UART, SPI, I$^2$C/TWI protocols
            \item Eagle and KiCad EDA software
            \item Soldering and prototyping, electronics workbench
        \end{itemize*}
    \item \textbf{RF Communications}
        \begin{itemize*}
            \item Digital Signal Processing (DSP) in GNU Radio and MATLAB
            \item Universal Software Radio Peripheral (USRP)
            \item Amateur Radio, General Class, KK4DOP
            \item Processing from feedpoint to baseband and demodulation
        \end{itemize*}
    \item \textbf{Peripheral Hardware}
        \begin{itemize*}
            \item Kinect, LIDAR, ultrasonic, IR, PIR
            \item Servos, DC motors, linear actuators, encoders
            \item 802.15.4 XBee radios
            \item Canon cameras and CHDK
        \end{itemize*}
    \item \textbf{Foreign Languages}
        \begin{itemize*}
            \item Latin: Two years of study
            \item Ancient (Attic) Greek: One year of study
            \item Spanish: Two years of study
        \end{itemize*}
    \item \textbf{Miscellaneous}
        \begin{itemize*}
            \item QuickBooks, GnuCash
            \item Detailed documentation practices
            \item Antarctic survival and igloo design
        \end{itemize*}
\end{itemize*}

\hrule
\vspace{-0.4em}
\subsection*{Projects}
    \begin{itemize}
        \item \headerrow{\textbf{Textplain}}{\emph{April 2015}}
        \begin{itemize}
            \item Built a personal website without a CMS using Django,
                PostgreSQL, and Nginx.  The site avoids the use of client-side
                javascript yet still follows a mobile-first design philosophy.
        \end{itemize}

        \item \headerrow{\textbf{Recipes}}{\emph{December 2014}}
        \begin{itemize}
            \item Built a MEAN stack database and client-side application for
                managing recipes with MongoDB, Node.js, and Angular.js.  The
                server allows family and friends to add, edit, and comment on
                recipes in one central location, rather than maintaining a long
                list of bookmarks or physical binder.  It's designed for mobile
                use and easy to use from the kitchen.
        \end{itemize}

        \item \headerrow{\textbf{Glorb}}{\emph{Spring 2013}}
        \begin{itemize}
            \item With my wife, designed and prototyped a science education
                project for MIT\@.  Glorbs are a whiffle ball containing a
                circuit with an AVR ATmega, an accelerometer, a lithium-ion
                battery pack, and a bunch of RGB LEDs.  The LEDs change color
                in response to acceleration---red for freefall, blue for high
                acceleration.  The goal is to explore physics and principles of
                acceleration by playing catch with them.  Glorbs give learners
                concrete visual feedback about the acceleration forces on a
                ball during the course of normal play.  We presented this at
                the MIT Media Lab's Other Festival in Spring 2013.
        \end{itemize}

        \item \headerrow{\textbf{Motion Lights}}{\emph{2013}}
        \begin{itemize}
            \item With spare PIR motion sensors and AVR chips, prototyped a
                power-efficient motion-triggered relay with AA batteries and
                tunable delay before shutoff.  This is small enough to fit in a
                junction box behind a lightswitch.  The lightswitch being on
                the wrong side of the kitchen is no longer an issue.
        \end{itemize}
    \end{itemize}

\hrule
\vspace{-0.4em}
\subsection*{Interests}
\begin{itemize*}
    \item Amateur Radio, General Class, KK4DOP
    \item Hiking, camping, photography, bread baking
    \item Accounting and tax law
    \item Intellectual property law
        %\item[Open Source Contributions:]
        %avr-osccal and other github stuff
\end{itemize*}
\end{document}
% vim:set spell:
